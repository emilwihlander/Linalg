\chapter{6}{Rummet $\mathbb{R}^n$}
\begin{task}{6.2 d)}
	\begin{align*}
		&x_1(1,2,2,0)+x_2(2,-1,0,2)+x_3(2,0,-1,-2)+x_4(0,2,-2,1)=0 \lra \\ \lra
		&(x_1+2x_2+2x_3,2x_1-x_2+2x_4,2x_1-x_3-2x_4,2x_2-2x_3+x_4)=0 \lra \\ \lra
		&\begin{cases}
			x_1+2x_2+2x_3=0 \\
			2x_1-x_2+2x_4=0 \\
			2x_1-x_3-2x_4=0 \\
			2x_2-2x_3+x_4=0
		\end{cases} \lra
		\begin{cases}
			x_1+2x_2+2x_3=0 \\
			-5x_2-4x_3+2x_4=0 \\
			-4x_2-5x_3-2x_4=0 \\
			2x_2-2x_3+x_4=0
		\end{cases} \lra
		\begin{cases}
			x_1+2x_2+2x_3=0 \\
			-5x_2-4x_3+2x_4=0 \\
			-9x_3-18x_4=0 \\
			45x_4=0
		\end{cases}
	\end{align*}
	
	\ans Vektorerna är linjärt oberoende
\end{task}