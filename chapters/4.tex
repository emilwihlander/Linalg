\chapter{4}{Skalärprodukt}
\begin{task}{4.1 a)}
	\[u\*v=4\*3\*\cos\frac{\pi}{4}=6\sqrt{2}\]

	\ans $6\sqrt{2}$
\end{task}

\begin{task}{b)}
	\begin{align*}
		&(u-2v)\*(3u+v)=
		3|u|^2+u\*v-2v\*3u-2|v|^2= \\ =
		&3\*4^2+6\sqrt{2}-6\*6\sqrt{2}-2\*3^2=
		30(1-\sqrt{2})
	\end{align*}

	\ans $30(1-\sqrt{2})$
\end{task}

\begin{task}{c)}
	\[(u+v)\*(u+v)=
	|u|^2+2u\*v+|v|^2=
	16+2\*6\sqrt{2}+9=
	25+12\sqrt(2)\]

	\ans $25+12\sqrt(2)$
\end{task}

\begin{task}{d)}
	\[|u+v|=\sqrt{(u+v)\*(u+v)}=\sqrt{25+12\sqrt(2)}\]

	\ans $\sqrt{25+12\sqrt(2)}$
\end{task}

\begin{task}{e)}
	\[(2u+3v)\*(2u-3v)=|2u|^2-|3v|^2=4|u|^2-9|v|^2=4\*16-9\*9=-17\]

	\ans $-17$
\end{task}

\begin{task}{4.2}
	\begin{align*}
		&|u+v|=|u-v| \lra
		|u+v|^2=|u-v|^2 \lra \\ \lra
		&(u+v)\*(u+v)=(u-v)\*(u-v) \lra \\ \lra
		&|u|^2+2u\*v+|v|^2=|u|^2-2u\*v+|v|^2 \lra \\ \lra
		&4u\*v=0 \lra
		u\*v=0 \ra
		\text{ $v$ och $u$ är ortogonala \proof}
	\end{align*}
	
	\ans
\end{task}

\begin{task}{4.3}
	\[(u-v)\*(2u+v)=
	2u\*u+u\*v-2v\*u-v\*v=
	2|u|^2-|v|^2 \mproof\]
\end{task}

\begin{task}{4.4 a)}
	$u+v$ och $u-v$.
\end{task}

\begin{task}{b)}
	$|u|=|v|$ enligt uppgift.
	\[(u+v)\*(u-v)=|u|^2-|v|^2=|u|^2-|u|^2=0 \mproof\]
\end{task}

\begin{task}{4.5}
	\[W=F*s\]
	\[s=(3,0,4)-(1,2,-1)=(2,-2,5)\]
	\[F=(1,-2,-1)\]
	\[W=(1,-2,-1)\*(-2,2,-5)=
	1\*(2)+(-2)\*(-2)+(-1)\*(5)=+2+4-5=1\]

	\ans $1Nm$
\end{task}

\begin{task}{4.6}
	$k=1$ ger:
	\[x_1e_1*e_1+x_2e_2*e_1+x_3e_3*e_1=1 \lra
	x_1=1\]
	$k=2$ ger:
	\[x_1e_1*e_2+x_2e_2*e_2+x_3e_3*e_2=2^2 \lra
	x_2=4\]
	$k=3$ ger:
	\[x_1e_1*e_3+x_2e_2*e_3+x_3e_3*e_3=3^2 \lra
	x_3=9\]
	\[u=(1,4,9)\]

	\ans $(1,4,9)$
\end{task}

\begin{task}{4.7 a)}
	Låt $\alpha=\frac{\pi}{2}-\beta$
	\begin{align*}
		&|(\cos\alpha,\cos\beta)|=
		\sqrt{\cos^2\alpha+\cos^2\beta}=
		\sqrt{\cos^2(\frac{\pi}{2}-\beta)+\cos^2\beta}= \\ =
		&\sqrt{\sin^2\beta+\cos^2\beta}=
		\sqrt{1}=1 \mproof
	\end{align*}
	
	\ans $\alpha+\beta=\frac{\pi}{2}$
\end{task}

\begin{task}{b)}
	KAN INTE LÖSA UPPGIFTEN
	
	Låt $\alpha+\beta+\gamma=\frac{\pi}{2}$
	\begin{align*}
		&|(\cos\alpha,\cos\beta,\cos\gamma)|=
		\sqrt{\cos^2\alpha+\cos^2\beta+\cos^2\gamma}=
		\sqrt{\cos^2(\frac{\pi}{2}-\beta)+\cos^2\beta}= \\ =
		&\sqrt{\sin^2\beta+\cos^2\beta}=
		\sqrt{1}=1 \mproof
	\end{align*}

	\ans $\alpha+\beta=\frac{\pi}{2}$
\end{task}

\pagebreak
\begin{task}{4.8}
	\[f_1\*f_2=\frac{1}{5}(3,4)\*\frac{1}{5}(-4,3)=\frac{3}{5}\*(-\frac{4}{5})+\frac{4}{5}\*\frac{3}{5}=0\]
	\[f_1\*f_1=\frac{1}{5}(3,4)\*\frac{1}{5}(3,4)=\frac{3}{5}\*\frac{3}{5}+\frac{4}{5}\*\frac{4}{5}=1\]
	\[f_2\*f_2=\frac{1}{5}(-4,3)\*\frac{1}{5}(-4,3)=(-\frac{4}{5})\*(-\frac{4}{5})+\frac{3}{5}\*\frac{3}{5}=1\]
	$f_1$ och $f_2$ utgör en ortonormerad bas i planet.
	\begin{align*}
		&x_1e_1+x_2e_2=y_1f_1+y_2f_2=
		y_1(\frac{3}{5}e_1+\frac{4}{5}e_2)+y_2(-\frac{4}{5}e_1+\frac{3}{5}e_2)= \\ =
		&(\frac{3}{5}y_1-\frac{4}{5}y_2)e_1+(\frac{4}{5}y_1+\frac{3}{5}y_2)e_2 \lra
		\begin{cases}
			x_1=\frac{3}{5}y_1-\frac{4}{5}y_2 \\
			x_2=\frac{4}{5}y_1+\frac{3}{5}y_2
		\end{cases}
	\end{align*}
	\[u=2e_1-e_2=
	\begin{cases}
		2=\frac{3}{5}y_1-\frac{4}{5}y_2 \\
		-1=\frac{4}{5}y_1+\frac{3}{5}y_2
	\end{cases} \lra
	\begin{cases}
		y_1=\frac{2}{5} \\
		y_2=-\frac{11}{5}
	\end{cases}
	\]
	$u$ i $f_1,f_2$ basen är $\frac{1}{5}(2,-11)$
	
	\ans $\frac{1}{5}(2,-11)$
\end{task}

\pagebreak
\begin{task}{4.9 a)}
	Om alla är vektorerna är ortogonala mot varandra och har längden 1 utgör de en ortonormerad bas.
	\[e^{'}_1\*e^{'}_2=\frac{1}{3}\*\frac{2}{3}+\frac{2}{3}\*\frac{1}{3}-\frac{2}{3}\*\frac{2}{3}=0\]
	\[e^{'}_2\*e^{'}_3=\frac{2}{3}\*\frac{2}{3}-\frac{1}{3}\*\frac{2}{3}-\frac{2}{3}\*\frac{1}{3}=0\]
	\[e^{'}_1\*e^{'}_3=\frac{1}{3}\*\frac{2}{3}-\frac{2}{3}\*\frac{2}{3}+\frac{2}{3}\*\frac{1}{3}=0\]
	\[e^{'}_1\*e^{'}_1=\left(\frac{1}{3}\right)^2+\left(\frac{2}{3}\right)^2+\left(-\frac{2}{3}\right)^2=1\]
	\[e^{'}_2\*e^{'}_2=\left(\frac{2}{3}\right)^2+\left(\frac{1}{3}\right)^2+\left(\frac{2}{3}\right)^2=1\]
	\[e^{'}_3\*e^{'}_3=\left(\frac{2}{3}\right)^2+\left(-\frac{2}{3}\right)^2+\left(-\frac{1}{3}\right)^2=1\]
	Vektorerna utger en ortonormerad bas.
	\[x_1e_1+x_2e_2+x_3e_3=x^{'}_1e^{'}_1+x^{'}_2e^{'}_2+x^{'}_3e^{'}_3\]
	\begin{align*}
		HL=&
		x^{'}_1(\frac{1}{3}e_1+\frac{2}{3}e_2-\frac{2}{3}e_3)+
		x^{'}_2(\frac{2}{3}e_1+\frac{1}{3}e_2+\frac{2}{3}e_3)+
		x^{'}_3(\frac{2}{3}e_1-\frac{2}{3}e_2-\frac{1}{3}e_3)= \\ =
		&(\frac{1}{3}x^{'}_1+\frac{2}{3}x^{'}_2+\frac{2}{3}x^{'}_3)e_1+
		(\frac{2}{3}x^{'}_1+\frac{1}{3}x^{'}_2-\frac{2}{3}x^{'}_3)e_2+
		(-\frac{2}{3}x^{'}_1+\frac{2}{3}x^{'}_2-\frac{1}{3}x^{'}_3)e_3
	\end{align*}
	Identifiera variablerna:
	\[\begin{cases}
		x_1=\frac{1}{3}x^{'}_1+\frac{2}{3}x^{'}_2+\frac{2}{3}x^{'}_3 \\
		x_2=\frac{2}{3}x^{'}_1+\frac{1}{3}x^{'}_2-\frac{2}{3}x^{'}_3 \\
		x_3=-\frac{2}{3}x^{'}_1+\frac{2}{3}x^{'}_2-\frac{1}{3}x^{'}_3
	\end{cases}\]
	Sätt in $u$:
	\begin{align*}
		&\begin{cases}
			1=\frac{1}{3}x^{'}_1+\frac{2}{3}x^{'}_2+\frac{2}{3}x^{'}_3 \\
			-1=\frac{2}{3}x^{'}_1+\frac{1}{3}x^{'}_2-\frac{2}{3}x^{'}_3 \\
			2=-\frac{2}{3}x^{'}_1+\frac{2}{3}x^{'}_2-\frac{1}{3}x^{'}_3
		\end{cases} \lra
		\begin{cases}
			3=x^{'}_1+2x^{'}_2+2x^{'}_3 \\
			-3=2x^{'}_1+x^{'}_2-2x^{'}_3 \\
			6=-2x^{'}_1+2x^{'}_2-x^{'}_3
		\end{cases} \lra
		\begin{cases}
			3=x^{'}_1+2x^{'}_2+2x^{'}_3 \\
			-9=-3x^{'}_2-6x^{'}_3 \\
			12=6x^{'}_2+3x^{'}_3
		\end{cases} \lra \\ \lra
		&\begin{cases}
			3=x^{'}_1+2x^{'}_2+2x^{'}_3 \\
			-9=-3x^{'}_2-6x^{'}_3 \\
			-6=-9x^{'}_3
		\end{cases} \lra
		\begin{cases}
			x_1=-\frac{5}{3} \\
			x_2=\frac{5}{3} \\
			x_3=\frac{2}{3}
		\end{cases}
	\end{align*}

	\ans $u=\frac{1}{3}(-5,5,1)$
\end{task}

\begin{task}{b)}
	\[u=\frac{1}{3}(x_1+2x_2-2x_3,2x_1+x_2+2x_3,2x_1-2x_2-x_3)\]
\end{task}

\begin{task}{4.10 a)}
	\[\cos[u,v]=\frac{u\*v}{|u||v|} \lra
	[u,v]=\arccos\frac{u\*v}{|u||v|}\]
	\begin{align*}
		[u,v]=&\arccos\frac{(1,\sqrt{3})\*(0,\sqrt{3})}{\sqrt{1^2+\sqrt{3}^2}\sqrt{\sqrt{3}^2}}=
		\arccos\frac{1\*0+\sqrt{3}\*\sqrt{3}}{2\sqrt{3}}= \\ =
		&\arccos\frac{3}{2\sqrt{3}}=
		\arccos\frac{\sqrt{3}}{2}=
		\frac{\pi}{6}
	\end{align*}

	\ans $\frac{\pi}{6}$
\end{task}

\begin{task}{b)}
	\[\cos[u,v]=\frac{u\*v}{|u||v|} \lra
	[u,v]=\arccos\frac{u\*v}{|u||v|}\]
	\begin{align*}
		[u,v]=&\arccos\frac{(\cos2,\sin2)\*(\cos3,\sin3)}{1\*1}=
		\arccos\cos(3-2)= 1
	\end{align*}

	\ans $\frac{\pi}{6}$
\end{task}

\begin{task}{4.11}
	Skapa vektorn $u=(-1,0,2)$ och $v=(-1,1,1)$.
	\[\cos[u,v]=\frac{u\*v}{|u||v|} \lra
	[u,v]=\arccos\frac{u\*v}{|u||v|}\]
	\begin{align*}
		[u,v]=&\arccos\frac{(-1,0,2)\*(-1,1,1)}{\sqrt{(-1)^2+2^2}\sqrt{(-1)^2+1^1+1^2}}= \\ =
		&\arccos\frac{-1\*(-1)+0\*1+2\*1}{\sqrt{5}\sqrt{3}}=
		\arccos\frac{3}{\sqrt{5}\sqrt{3}}=
		\arccos\sqrt{\frac{3}{5}}
	\end{align*}

	\ans $\arccos\sqrt{0.6}$
\end{task}

\begin{task}{4.12}
	Eftersom triangeln är liksidig är $[u,v]=\pi/3$ och $|u|=|v|=a$.
	\[\cos[u,v]=\frac{u\*v}{|u||v|} \lra
	[u,v]=\arccos\frac{u\*v}{|u||v|}\]
	\begin{align*}
		&|2u+v||3v-2u|=
		\sqrt{(2u+v)\*(2u+v)}\sqrt{(3v-2u)\*(3v-2u)}= \\ =
		&\sqrt{4u\*u+4u\*v\*\cos\frac{\pi}{3}+v\*v}\sqrt{9v\*v-6u\*v\*\cos\frac{\pi}{3}+u\*u}= \\ =
		&\sqrt{4a^2+2a^2+a^2}\sqrt{9a^2-3a^2+a^2}=
		\sqrt{7a^2}\sqrt{7a^2}=
		7a^2
	\end{align*}
	\begin{align*}
		&[2u+v,3v-2u]=\arccos\frac{(2u+v)\*(3v-2u)}{|2u+v||3v-2u|}= \\ =
		&\arccos\frac{6u\*v-4u\*u+3v\*v-2u\*v}{7a^2}= \\ =
		&\arccos\frac{4a^2\cos\frac{\pi}{3}-4a^2+3a^2}{7a^2}=
		\arccos\frac{a^2}{7a^2}=
		\arccos\frac{1}{7}
	\end{align*}

	\ans $\arccos\frac{1}{7}$
\end{task}

\begin{task}{4.13}
	Låt $u=(1,0,2)-(0,-1,1)=(1,1,1)$ och $v=(2,1,2)-(0,-1,1)=(2,2,1)$
	\begin{align*}
		[u,v]=
		\arccos\frac{u\*v}{|u||v|}=
		\arccos\frac{(1,1,1)\*(2,2,1)}{\sqrt{1+1+1}\sqrt{2^2+2^2+1}}=
		\arccos\frac{2+2+1}{\sqrt{3}3}=
		\arccos\frac{5}{3\sqrt{3}}
	\end{align*}
	Låt $u=(0,-1,1)-(1,0,2)=(-1,-1,-1)$ och $v=(2,1,2)-(1,0,2)=(1,1,0)$
	\begin{align*}
		[u,v]=
		\arccos\frac{u\*v}{|u||v|}=
		\arccos\frac{(-1,-1,-1)\*(1,1,0)}{\sqrt{1+1+1}\sqrt{1+1}}=
		\arccos\frac{-1-1}{\sqrt{3}\sqrt{2}}=
		\arccos\sqrt{-\frac{2}{3}}
	\end{align*}
	Låt $u=(0,-1,1)-(2,1,2)=(-2,-2,-1)$ och $v=(1,0,2)-(2,1,2)=(-1,-1,0)$
	\begin{align*}
		[u,v]=
		\arccos\frac{u\*v}{|u||v|}=
		\arccos\frac{(-2,-2,-1)\*(-1,-1,0)}{\sqrt{2^2+2^2+1}\sqrt{1+1}}=
		\arccos\frac{2+2}{\sqrt{3}\sqrt{2}}=
		\arccos\frac{2\sqrt{2}}{\sqrt{3}}
	\end{align*}

	\ans Sidorna $3$, $\sqrt{3}$ och $\sqrt{2}$. vinklarna $\arccos\frac{5}{3\sqrt{3}}$, $\arccos\sqrt{-\frac{2}{3}}$ och $\arccos\frac{2\sqrt{2}}{\sqrt{3}}$
\end{task}

\begin{task}{4.14}
	Låt tetraedern ha sina hörn i $(2,0,0)$, $(0,2,0)$, $(0,0,2)$ och $(2,2,2)$. Då ligger kolatomen i $(1,1,1)$.
	Låt $u = (1,1,1)-(2,0,0) = (1,-1,-1)$ och $v= (1,1,1)-(0,2,0) = (-1,1,-1)$
	\begin{align*}
		&[u,v] = \arccos\frac{u\*v}{|u||v|} =
		\arccos\frac{(1,-1,-1)\*(-1,1,-1)}{\sqrt{1^2+(-1)^2+(-1)^2}\sqrt{(-1)^2+1^2+(-1)^2}}= \\ =
		&\arccos\frac{-1-1+1}{\sqrt{3}\sqrt{3}}=
		\arccos\left(-\frac{1}{3}\right)
	\end{align*}

	\ans $\arccos\left(-\frac{1}{3}\right)$
\end{task}

\begin{task}{4.15}
	Låt $u=(x,y,z)$
	\[\cos\frac{\pi}{2}=\frac{(x,y,z)\*(1,0,1)}{\sqrt{2}} \lra
	x+z=0\]
	\[\cos\frac{\pi}{2}=\frac{(x,y,z)\*(1,2,3)}{\sqrt{14}} \lra
	x+2y+3z=0\]
	Låt $z=t$
	\[\begin{cases}
		x+2y+3z=0 \\
		x+z=0
	\end{cases} \lra
	\begin{cases}
		x=-t
		y=-t
		z=t
	\end{cases} \]
	Att längden ska vara 1 ger:
	\[x^2+y^2+z^2=1 \lra
	(-t)^2+(-t)^2+t^2=1 \lra
	3t^2=1 \lra
	t=\pm\frac{1}{\sqrt{3}}\]
	Vilket ger vektorerna:
	\[\pm\frac{1}{\sqrt{3}}(1,1,-1)\]

	\ans $\pm\frac{1}{\sqrt{3}}(1,1,-1)$
\end{task}

\pagebreak
\begin{task}{4.16}
	Låt $v_3=(x,y,z)$
	För att det ska vara en ortonormerad bas ska:
	\[v_3\*v_2=0,~~v_3*v_1=0,~~v_3*v_3=1\]
	\[v_3\*v_2=(x,y,z)*\frac{1}{3}(1,2,2)=\frac{1}{3}x+\frac{2}{3}y+\frac{2}{3}z=0\]
	\[v_3\*v_1=(x,y,z)*\frac{1}{3}(-2,-1,2)=-\frac{2}{3}x-\frac{1}{3}y+\frac{2}{3}z=0\]
	\[v_3\*v_3=x^2+y^2+z^2=1\]
	Vi väntar med $v_3\*v_3$ tillsvidare och löser nedanstående ekvationssystem. Låt $z=t$
	\[\begin{cases}
		\frac{1}{3}x+\frac{2}{3}y+\frac{2}{3}z=0 \\
		-\frac{2}{3}x-\frac{1}{3}y+\frac{2}{3}z=0
	\end{cases} \lra
	\begin{cases}
		\frac{1}{3}x+\frac{2}{3}y+\frac{2}{3}z=0 \\
		y+2z=0
	\end{cases} \lra
	\begin{cases}
		x=2t
		y=-2t
		z=t
	\end{cases}\]
	Sätt in värdena i sista formeln:
	\[(2t)^2+(-2t)^2+t^2=1 \lra
	9t^2=1 \lra
	t=\pm\frac{1}{3}\]
	Vilket ger sista vektorn:
	\[v_3=\frac{1}{3}(-2,2,-1) \text{ eller } v_3=\frac{1}{3}(2,-2,1)\]
	Bestäm $u$:
	\[x_1e_1+x_2e_2+x_3e_3=y_1v_1+y_2v_2+y_3v_3\]
	\begin{align*}
		HL=&y_1\frac{1}{3}(-2e_1-e_2+2e_3)+y_2\frac{1}{3}(e_1+2e_2+2e_3)+y_3\frac{1}{3}(-2e_1+2e_2-e_3)= \\ =
		&\frac{1}{3}(-2y_1+y_2-2y_3)e_1+\frac{1}{3}(-y_1+2y_2+2y_3)e_2+\frac{1}{3}(2y_1+2y_2-y_3)e_3
	\end{align*}
	matcha och sätt in värdena:
	\begin{align*}
		&\begin{cases}
			1=\frac{1}{3}(-2y_1+y_2-2y_3) \\
			1=\frac{1}{3}(-y_1+2y_2+2y_3) \\
			1=\frac{1}{3}(2y_1+2y_2-y_3)
		\end{cases} \lra
		\begin{cases}
			3=-2y_1+y_2-2y_3 \\
			3=3y_2+6y_3 \\
			6=3y_2-3y_3 
		\end{cases} \lra \\ \lra
		&\begin{cases}
			3=-2y_1+y_2-2y_3 \\
			3=3y_2+6y_3 \\
			3=-9y_3 
		\end{cases} \lra
		\begin{cases}
			y_1=-\frac{1}{3} \\
			y_2=\frac{5}{3} \\
			y_3=-\frac{1}{3}
		\end{cases}
	\end{align*}
	\[u=\frac{1}{3}(-1,5,-1)\]

	\ans $\frac{1}{3}(-1,5,-1)$
\end{task}

\pagebreak
\begin{task}{4.17}
	Låt $z=t$
	\[\begin{cases}
		x+y+3z+6=0 \\
		2x+y-2z-10=0
	\end{cases} \lra
	\begin{cases}
		x+y+3z+6=0 \\
		-y-8z-22=0
	\end{cases} \lra
	\begin{cases}
		x=5t+16
		y=-8t-22
		z=t
	\end{cases}\]
	\[L_1=(5t+16,-8t-22,t)\]
	Riktning:
	\[(6,0,1)-(-4,4,-7)=(10,-4,8)=2(5,-2,4)\]
	Godtycklig riktning: $(5,-2,4)$
	\[\begin{cases}
		x=-4+5t \\
		y=4-2t \\
		z=-7+4t
	\end{cases}\]
	Linjerna skär varandra om det finns ett $t_1$ och $t_2$ så att:
	\[\begin{cases}
		t_1 = -7+4t_2 \\
		5t_1+16 = -4+5t_2 \\
		-8t_1-22 = 4-2t_2
	\end{cases} \lra
	\begin{cases}
		t_1 = -7+4t_2 \\
		16 = 31-15t_2 \\
		-22 = -52+30t_2
	\end{cases} \lra
	\begin{cases}
		t_1=-3
		t_2=1
	\end{cases}\]
	Linjerna skär varandra i punkten $(1,2,-3)$.
	Skapa två vektorer från linjerna:
	\[v_1=(6,-6,-2)-(1,2,-3)=(5,-8,1)\]
	\[v_2=(6,0,1)-(1,2,-3)=(5,-2,4)\]
	\begin{align*}
		\cos[v_1,v_2]=&
		\frac{(5,-8,1)\*(5,-2,4)}{\sqrt{5^2+8^2+1^2}\sqrt{5^2+2^2+4^2}}=
		\frac{5\*5+8\*2+4}{\sqrt{90}\sqrt{45}}=
		\frac{45}{45\sqrt{2}}=
		\frac{1}{\sqrt{2}} \lra \\ \lra
		[v_1,v_2]=&\arccos\frac{1}{\sqrt{2}}=\frac{\pi}{4}
	\end{align*}

	\ans vinkeln $\pi/4$ och skärningspunkten $(1,2,-3)$
\end{task}

\begin{task}{4.18 a)}
	\[-3\*1+4\*2=5\]
	\[-3x+4y-5=0\]

	\ans $-3x+4y-5=0$
\end{task}

\begin{task}{b)}
	\[-2+1=-1\]
	\[-2x+3y+z+1=0\]

	\ans $-2x+3y+z+1=0$
\end{task}

\pagebreak
\begin{task}{4.19}
	\[\begin{cases}
		2x+y-1=0 \\
		3x+4y-3=0
	\end{cases} \lra
	\begin{cases}
		2x+y-1=0 \\
		-5x+1=0
	\end{cases} \lra
	\begin{cases}
		x=\frac{1}{5} \\
		y=\frac{3}{5}
	\end{cases}\]
	Skärningspunkt: $(\frac{1}{5}, \frac{3}{5})$

	Skapa vektorer av linjerna.
	\[v_1=(\frac{6}{5},-\frac{7}{5})-(\frac{1}{5}, \frac{3}{5})=(1,-2)\]
	\[v_2=(\frac{21}{5}, -\frac{12}{5})-(\frac{1}{5}, \frac{3}{5})=(4,-3)\]
	Riktningsvektorerna som används är $(1,-2)$ och $(4,-3)$.
	\begin{align*}
		\cos[v_1,v_2]=&
		\frac{(1,-2)\*(4,-3)}{\sqrt{1^2+2^2}\sqrt{4^2+3^2}}=
		\frac{4+3\*2}{5\sqrt{5}}=
		\frac{2}{\sqrt{5}} \lra \\ \lra
		[v_1,v_2]=&\arccos\frac{2}{\sqrt{5}}
	\end{align*}

	\ans Skärningspunkt: $(\frac{1}{5}, \frac{3}{5})$, vinkel: $\arccos\frac{2}{\sqrt{5}}$
\end{task}

\begin{task}{4.20}
	Planens normalriktning: $v_1=(2,1,-1)$ och $v_2=(1,-3,2)$.
	\begin{align*}
		\cos[v_1,v_2]=&
		\frac{(2,1,-1)\*(1,-3,2)}{\sqrt{2^2+1^2+1^2}\sqrt{1^2+3^2+2^2}}=
		\frac{2-3-2}{2\sqrt{21}}=
		\frac{-3}{2\sqrt{21}} \lra \\ \lra
		[v_1,v_2]=&\arccos\frac{-3}{2\sqrt{21}}
	\end{align*}

	\ans $\arccos\frac{-3}{2\sqrt{21}}$
\end{task}

\begin{task}{4.21 a)}
	\ans
\end{task}

\begin{task}{b)}
	\ans
\end{task}

\begin{task}{4.22 a)}
	Låt $v_1$ och $v_2$ vara vinkelräta komposanter till $u$. $v_1$ är parallell med vektorn $(1,4,0)$.
	\[v_1=\frac{(1,-1,1)\*(1,4,0)}{|(1,4,0)|^2}(1,4,0)=
	\frac{1-4}{1^2+4^2}(1,4,0)=
	-\frac{3}{17}(1,4,0)\]
	\[v_2=u-v_1=(1,-1,1)+\frac{3}{17}(1,4,0)=\frac{1}{17}(20,-5,17)\]

	\ans $v_1=-\frac{3}{17}(1,4,0)$ och $v_2=\frac{1}{17}(20,-5,17)$
\end{task}

\pagebreak
\begin{task}{b)}
	Låt $y=t$
	\[\begin{cases}
		x=4t+1 \\
		y=t \\
		z=2t-1
	\end{cases}\]
	Skapa en riktningsvektor genom att välja två slumpmässiga punkter på linjen beräkna vektorn däremellan.
	Punkterna: $t=0 \ra (1,0-1)$ och $t=1 \ra (5,1,1)$ . Riktningsvektorn:
	\[v = (5,1,1)-(1,0,-1)=(4,1,2)\]
	Låt $v_1$ och $v_2$ vara vinkelräta komposanter till $u$. $v_1$ är parallell med vektorn $(4,1,2)$.
	\[v_1=\frac{(1,-1,1)\*(4,1,2)}{|(4,1,2)|^2}(4,1,2)=
	\frac{4-1+2}{4^2+1^2+2^2}(4,1,2)=
	\frac{5}{21}(4,1,2)\]
	\[v_2=u-v_1=
	(1,-1,1)-\frac{5}{21}(4,1,2)=
	\frac{1}{21}(1,-26,11)\]

	\ans $v_1=\frac{5}{21}(4,1,2)$ och $v_2=\frac{1}{21}(1,-26,11)$
\end{task}

\begin{task}{c)}
	Låt $v_1$ och $v_2$ vara vinkelräta komposanter till $u$. $v_2$ är parallell med det givna planet.
	Planets normalriktning är $(1,2,-1)$.
	\[v_1=\frac{(1,-1,1)\*(1,2,-1)}{|(1,2,-1)|^2}(1,2,-1)=
	\frac{1-2-1}{1^2+2^2+1^2}(1,2,-1)=
	-\frac{1}{3}(1,2,-1)\]
	\[v_2=u-v_1=(1,-1,1)+\frac{1}{3}(1,2,-1)=\frac{1}{3}(4,-1,2)\]

	\ans $v_1=-\frac{1}{3}(1,2,-1)$ och $v_2=\frac{1}{3}(4,-1,2)$
\end{task}

\begin{task}{4.23}
	\ans
\end{task}

\begin{task}{4.24}
	\ans
\end{task}

\begin{task}{4.25 a)}
	Planets normalriktning är $(1,1,-1)$.
	Välj slumpmässig punkt i planet: $(1,1,0)$.
	Vektorn mellan punkterna: 
	\[(1,0,1)-(1,1,0)=(0,-1,1)\]
	Låt $v$ vara det sökta avståndet.
	\[v=\frac{(0,-1,1)\*(1,1,-1)}{|(1,1,-1)|^2}(1,1,-1)=
	\frac{-1-1}{1^2+1^2+1^2}(1,1,-1)=
	-\frac{2}{3}(1,1,-1)\]
	\[|v|=\sqrt{\frac{2^2}{3^2}+\frac{2^2}{3^2}+\frac{2^2}{3^2}}=
	\sqrt{\frac{4}{3}}=
	\frac{2}{\sqrt{3}}\]

	\ans $\frac{2}{\sqrt{3}}$
\end{task}

\begin{task}{b)}
	\ans
\end{task}

\begin{task}{c)}
	\ans
\end{task}

\begin{task}{4.26}
	\ans
\end{task}

\begin{task}{4.27 a)}
	\ans
\end{task}

\begin{task}{b)}
	\ans
\end{task}

\begin{task}{4.28}
	\ans
\end{task}

\begin{task}{4.29 a)}
	Skapa en riktningsvektor, $v$, genom att välj två punkter på linjen: $t=0 \ra (1,2,0)$ och $t=1 \ra (2,1,2)$.
	\[v=(2,1,2)-(1,2,0)=(1,-1,2)\]
	Låt $u$ vara vektorn mellan en punkt på linjen och punkten.
	\[u=(1,-1,1)-(1,2,0)=(0,-3,1)\]
	Låt $v_1$ och $v_2$ vara komposant vektorerna till $u$. $v_1$ är parallell med linjen och $v_2$ är det sökta avståndet.
	\[v_1=\frac{u\*v}{|v|^2}v=
	\frac{(0,-3,1)\*(1,-1,2)}{|(1,-1,2)|^2}(1,-1,2)=
	\frac{3+2}{1^2+1^2+2^2}(1,-1,2)=
	\frac{5}{6}(1,-1,2)\]
	\[v_2=u-v_1=
	(0,-3,1)-\frac{5}{6}(1,-1,2)=
	\frac{1}{6}(-5,-13,-4)\]
	\[|v_2|=\sqrt{\frac{5^2}{6^2}+\frac{13^2}{6^2}+\frac{4^2}{6^2}}=
	\sqrt{\frac{210}{36}}=
	\sqrt{\frac{35}{6}}\]

	\ans $\sqrt{\frac{35}{6}}$
\end{task}

\begin{task}{b)}
	\ans
\end{task}

\begin{task}{4.30 a)}
	\ans
\end{task}

\begin{task}{b)}
	\ans
\end{task}

\begin{task}{4.31 a)}
	Planets normalriktning: $(2,-1,2)$.
	Avståndet mellan planet och punkten:
	\[\frac{(1,0,1)\*(2,-1,2)}{|(2,-1,2)|^2}(2,-1,2)=
	\frac{2+2}{2^2+1^2+2^2}(2,-1,2)=
	\frac{4}{9}(2,-1,2)\]
	Projektionspunkten:
	\[P:(1,0,1)-\frac{4}{9}(2,-1,2)=\frac{1}{9}(1,4,1)\]

	\ans $\frac{1}{9}(1,4,1)$
\end{task}

\begin{task}{b)}
	Låt $u=(1,0,1)$ och $v=\frac{4}{9}(2,-1,2)$.
	Spegelbilden:
	\[P:u-2v=(1,0,1)-\frac{8}{9}(2,-1,2)=
	\frac{1}{9}(-7,8,-7)\]

	\ans $\frac{1}{9}(-7,8,-7)$
\end{task}

\begin{task}{4.32}
	\ans
\end{task}

\begin{task}{4.33}
	\ans
\end{task}

\begin{task}{4.34 a)}
	\ans
\end{task}

\begin{task}{b)}
	\ans
\end{task}

\begin{task}{4.35}
	\ans
\end{task}

\begin{task}{4.36}
	\ans
\end{task}

\begin{task}{4.37}
	\ans
\end{task}

\begin{task}{4.38}
	\ans
\end{task}

\begin{task}{4.39}
	\ans
\end{task}

\begin{task}{4.40}
	\ans
\end{task}

\begin{task}{4.41}
	\ans
\end{task}

\begin{task}{4.42}
	En punkt i planet:
	\[\frac{(1,3,4)+(3,-1,8)}{2}=(2,1,6)\]
	Normalriktning för planet:
	\[(3,-1,8)-(1,3,4)=2(1,-2,2)\]
	Ekvation för planet:
	\[x-2y+2z=d\]
	Sätt in punkt från planet:
	\[d=2-2+2\*6=12\]

	\ans $x-2y+2z=12$
\end{task}

\begin{task}{4.43}
	\ans
\end{task}

\begin{task}{4.44}
	\ans
\end{task}

\begin{task}{4.45}
	\ans
\end{task}

\begin{task}{4.46}
	\ans
\end{task}

\begin{task}{4.47}
	\ans
\end{task}

\begin{task}{4.48}
	\ans
\end{task}

\begin{task}{4.49}
	\ans
\end{task}