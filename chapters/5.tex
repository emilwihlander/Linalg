\chapter{5}{Vektorprodukt}
\begin{task}{5.1 a)}
	\[e_1\times e_3=-e_2\]
	\[e_3 \times e_2=-e_1\]
	
	\ans $-e_2$ och $-e_1$
\end{task}

\begin{task}{b)}
	\[e_2\times(2e_1-e_2+3e_3)=
	2e_2\times e_1-e_2\times e_2 + 3e_2\times e_3=
	-2e_3-0+3e_1=
	3e_1-2e_3\]
	
	\ans $3e_1-2e_3$
\end{task}

\begin{task}{c)}
	\begin{align*}
	&(1,0,1)\times(1,2,3)=
	\left(\begin{detmat}
	0 & 1 \\
	2 & 3 
	\end{detmat},
	-\begin{detmat}
	1 & 1 \\
	1 & 3
	\end{detmat},
	\begin{detmat}
	1 & 0 \\
	1 & 2
	\end{detmat}\right)= \\ =
	&(0\*3-1\*2,-(1\*3-1\*1),1\*2-0\*1)=
	(-2,-2,2)
	\end{align*}
	
	\ans $(-2,-2,2)$
\end{task}

\begin{task}{d)}
	\begin{align*}
	&(2,6,-3)\times(0,2,3)=
	\left(\begin{detmat}
	6 &-3 \\
	2 & 3 
	\end{detmat},
	-\begin{detmat}
	2 &-3 \\
	0 & 3
	\end{detmat},
	\begin{detmat}
	2 & 6 \\
	0 & 2
	\end{detmat}\right)= \\ =
	&(6\*3+3\*2,-(2\*3+3\*0),2\*2-6\*0)=
	(24,-6,4)
	\end{align*}
	
	\ans $2(12,-3,2)$
\end{task}

\begin{task}{5.2}
	\ans
\end{task}

\begin{task}{5.3}
	\ans
\end{task}

\begin{task}{5.4}
	\begin{align*}
	&(u+v)\*[(v+w)\times(w+u)]=
	(u+v)\*[v\times(w+u)+w\times(w+u)]= \\ =
	&(u+v)\*[v \times w + v \times u + w \times w + w \times u]= \\ =
	&u\*v \times w + u\*v \times u + u\*w \times u + v\*v \times w + v\*v \times u + v\*w \times u= \\ =
	&u\*v \times w + v\*w \times u =
	2u\*v \times w
	\end{align*}
	
	\ans $2u\*v \times w$
\end{task}

\begin{task}{5.5}
	\ans
\end{task}

\begin{task}{5.6 a)}
	\ans
\end{task}

\begin{task}{b)}
	\ans
\end{task}

\begin{task}{5.7}
	Skapa två vektorer som börjar i samma punkt av punkterna.
	\[v_1=(2,3,4)-(0,-1,1)=(2,4,3)\]
	\[v_2=(3,-1,2)-(0,-1,1)=(3,0,1)\]
	Arean av en triangel är hälften av arean på motsvarande parallellogram.
	\begin{align*}
		A=&\abs{\frac{1}{2}(2,4,3)\times(3,0,1)}=
		\frac{1}{2}\abs{
			\left(\begin{detmat}
				4 & 3 \\
				0 & 1
			\end{detmat},
			-\begin{detmat}
				2 & 3 \\
				3 & 1
			\end{detmat},
			\begin{detmat}
				2 & 4 \\
				3 & 0
			\end{detmat}\right)} = \\ =
		&\frac{1}{2}|(4\*1-3\*0,-(2\*1-3\*3),2\*0-4\*3)|=
		\frac{1}{2}|(4,7,-12)|= \\ =
		&\frac{1}{2}\sqrt{4^2+7^2+12^2}=
		\frac{\sqrt{209}}{2}
	\end{align*}
	
	\ans $\frac{\sqrt{209}}{2}$
\end{task}

\begin{task}{5.8 a)}
	\ans
\end{task}

\begin{task}{b)}
	\ans
\end{task}

\begin{task}{5.9}
	Skapa ett plan som är parallellt med både $\ell_1$ och $\ell_2$ och som sammanfaller med $\ell_1$.
	Skapa två riktningsvektorer från linjerna.
	
	två punkter på $\ell_1$: $t=0 \Rightarrow (3,1,3)$ och $t=1 \Rightarrow (4,0,6)$ \\
	två punkter på $\ell_2$: $t=0 \Rightarrow (2,-1,-3)$ och $t=1 \Rightarrow (3,0,-7)$ \\
	\[v_1=(4,0,6)-(3,1,3)=(1,-1,3)\]
	\[v_2=(3,0,-7)-(2,-1,-3)=(1,1,-4)\]
	Hitta normalriktningen genom att ta vektorprodukten av $v_1$ och $v_2$.
	\begin{align*}
		v_1\times v_2=&
		(1,-1,3)\times(1,1,-4)=
		\left(\begin{detmat}
			-1& 3 \\
			1 &-4
		\end{detmat},
		-\begin{detmat}
			1 & 3 \\
			1 &-4
		\end{detmat},
		\begin{detmat}
			1 &-1 \\
			1 & 1
		\end{detmat}\right) = \\ =
		&(1\*4-3\*1,-(-1\*4-3\*1),1\*1+1\*1)=
		(1,7,2)
	\end{align*}
	Vektor mellan planet och $\ell_2$:
	\begin{align*}
		v_3=&\frac{((2,-1,-3)-(3,1,3))*(1,7,2)}{|(1,7,2)|^2}(1,7,2)=
		\frac{-1\*1-2\*7-6\*2}{1^2+7^2+2^2}(1,7,2) = \\ =
		&\frac{-27}{54}(1,7,2)=
		-\frac{1}{2}(1,7,2)
	\end{align*}
	\[|v_3|=\sqrt{\frac{1^2+7^2+2^2}{2^2}}=\frac{1}{2}\sqrt{54}=\frac{3}{2}\sqrt{6}\]
	
	\ans $\frac{3}{2}\sqrt{6}$
\end{task}

\begin{task}{5.10}
	\ans
\end{task}

\begin{task}{5.11 a)}
	\ans
\end{task}

\begin{task}{b)}
	\ans
\end{task}

\begin{task}{5.12 a)}
	\ans
\end{task}

\begin{task}{b)}
	\ans
\end{task}

\begin{task}{5.13}
	\ans
\end{task}

\begin{task}{5.14 a)}
	\ans
\end{task}

\begin{task}{b)}
	\ans
\end{task}

\begin{task}{c)}
	\ans
\end{task}

\begin{task}{5.15}
	Hitta ett $k$ så att $\hat{e}_1=k(1,1,-1)$ och $|\hat{e}_1|=1$
	\[|\hat{e}_1
	|=1 \lra
	\sqrt{k^2+k^2+k^2}=1 \lra
	3k^2=1 \lra
	k=\pm\frac{1}{\sqrt{3}}\]
	Väljer $k=\frac{1}{\sqrt{3}}$.
	\[\hat{e}_1=\frac{1}{\sqrt{3}}(1,1,-1)\]
	för att hitta $\hat{e}_2=(x,y,z)$ hitta en lösning ekvationssystemet:
	\begin{align*}
		\begin{cases}
			(x,y,z)\*(1,1,-1)=0 \\
			x+y+z=0 \\
			x^2+y^2+z^2=1
		\end{cases} \lra
		\begin{cases}
			x+y-z=0 \\
			x+y+z=0 \\
			x^2+y^2+z^2=1
		\end{cases} \lra
		\begin{cases}
			x=\frac{1}{\sqrt{2}} \\
			y=-\frac{1}{\sqrt{2}} \\
			z=0
		\end{cases}
	\end{align*}
	$\hat{e}_3=\hat{e}_1\times \hat{e}_2$
	\begin{align*}
		\hat{e}_3=&
		\frac{1}{\sqrt{3}}(1,1,-1)\times \frac{1}{\sqrt{2}}(1,-1,0)= \\ =
		&\frac{1}{\sqrt{6}}\left(\begin{detmat}
		1 &-1 \\
		-1& 0 
		\end{detmat},
		-\begin{detmat}
		1 &-1 \\
		1 & 0
		\end{detmat},
		\begin{detmat}
		1 & 1 \\
		1 &-1
		\end{detmat}\right)= \\ =
		&\frac{1}{\sqrt{6}}(1\*0-1\*1,-(1\*0+1\*1),-1\*1-1\*1)=
		-\frac{1}{\sqrt{6}}(1,1,2)
	\end{align*}
	Normalriktningen till planet är $(1,1,1)$.
	\[(1,1,1)=\hat{a}_1\hat{e}_1+\hat{a}_2\hat{e}_2+\hat{a}_3\hat{e}_3\]
	\[\hat{a}_1=(1,1,1)*\hat{e}_1=(1,1,1)\*\frac{1}{\sqrt{3}}(1,1,-1)=\frac{1}{\sqrt{3}}\]
	\[\hat{a}_2=(1,1,1)*\hat{e}_2=(1,1,1)\*\frac{1}{\sqrt{2}}(1,-1,0)=0\]
	\[\hat{a}_3=(1,1,1)*\hat{e}_3=(1,1,1)\*(-\frac{1}{\sqrt{6}}(1,1,2))=-\frac{4}{\sqrt{6}}\]
	\[\pi:\frac{1}{\sqrt{3}}\hat{x}_1-\frac{4}{\sqrt{6}}\hat{x}_3=0 \lra
	\hat{x}_1-2\sqrt{2}\hat{x}_3=0\]
	
	\ans $\hat{e}_1=\frac{1}{\sqrt{3}}(1,1,-1)$, $\hat{e}_2=\frac{1}{\sqrt{2}}(1,-1,0)$, $\hat{e}_3=-\frac{1}{\sqrt{6}}(1,1,2)$ och $\pi:\hat{x}_1-2\sqrt{2}\hat{x}_3=0$
\end{task}

\begin{task}{5.16}
	\ans
\end{task}

\begin{task}{5.17}
	Låt $u=(x,y,z)$.
	\begin{align*}
		&(x,y,z)\times(1,a,-1)=(1,2,3) \lra \\ \lra
		&\left(\begin{detmat}
			y & z \\
			a &-1 
		\end{detmat},
		-\begin{detmat}
			x & z \\
			1 &-1
		\end{detmat},
		\begin{detmat}
			x & y \\
			1 & a
		\end{detmat}\right)=(1,2,3) \lra \\ \lra
		&(-y-az,-(-x-z),ax-y)=(1,2,3) \lra \\ \lra
		&\begin{cases}
			-y-az=1 \\
			x+z=2 \\
			ax-y=3
		\end{cases}
	\end{align*}
	låt $z=t$.
	\[\begin{cases}
		x=2-t \\
		y=2t-3 \\
		z=t \\
		a=\frac{2}{2}=1
	\end{cases}\]
\end{task}

\begin{task}{5.18}
	\ans
\end{task}

\begin{task}{5.19 a)}
	\ans
\end{task}

\begin{task}{b)}
	\ans
\end{task}

\begin{task}{5.20}
	\ans
\end{task}

\begin{task}{2.21}
	\ans
\end{task}

\begin{task}{5.22}
	\ans
\end{task}

\begin{task}{5.23 a)}
	\ans
\end{task}

\begin{task}{b)}
	\ans
\end{task}

\begin{task}{5.24}
	\ans
\end{task}

\begin{task}{5.25}
	\ans
\end{task}

\begin{task}{5.26}
	\ans
\end{task}