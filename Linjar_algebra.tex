% !TeX spellcheck = sv_SE
%http://www.cs.put.poznan.pl/ksiek/latexmath.html
%https://en.wikibooks.org/wiki/LaTeX/Advanced_Mathematics
%http://www.maths.lth.se/matematiklth/personal/magnusa/kurser/endim-ht2015/B1/kurspmB1ht15.pdf

\documentclass[a4paper]{article} 
\usepackage[T1]{fontenc} 
\usepackage[utf8]{inputenc} 
\usepackage[swedish]{babel} 
\usepackage{amsmath}
\usepackage{amssymb}
\usepackage{cancel}
\usepackage{graphicx}
\usepackage{systeme}

\usepackage{color}
\definecolor{tskcol}{RGB}{193,70,70}
\newcommand{\tskcol}[1]{\textcolor{tskcol}{#1}}

\newcommand\varpm{\mathbin{\vcenter{\hbox{%
				\oalign{\hfil$\scriptstyle+$\hfil\cr
					\noalign{\kern-.5ex}					
					$\scriptscriptstyle({-})$\cr}%
			}}}}
			
			\setlength{\parindent}{0in}
			

\title{Linjär algebra\\ FMA420} 
\author{Emil Wihlander\\ dat15ewi@student.lu.se} 
\date{2016--05--12}
\begin{document} 
\maketitle
\pagebreak
\section*{Kapitel 1: Linjära ekvationssystem}
\texttt{\tskcol{1.1~~~~~ (s.)}}
\begin{quotation}
	\noindent
	Börja nerifrån och upp och lös en variabel i taget.
	\begin{align*}
	&\systeme{
		2x+3y- z=5,
		  -3y+5z=1,
		      4z=8}  \\ \Leftrightarrow
	&\begin{cases}
	z=2 \\
	y=\frac{1-5*2}{-3}=3 \\
	x=\frac{5+2-3*3}{2}=-1
	\end{cases}
	\end{align*}
	\\
	\textbf{Svar}: $(x,y,z)=(-1,3,2)$
\end{quotation}

\texttt{\tskcol{1.2~~~~~ (s.)}}
\begin{quotation}
	\noindent
	Gausselimination
	\begin{align*}
	&\systeme{
		  x-2y+  z=2,
		 2x-6y+11z=35,
		-3x+5y+  z=8}  
	&\begin{array}{l} 
		(a) \\ 
		(b) \\
		(c)
	\end{array} \\ \Leftrightarrow
	&\systeme{
		x-2y+ z=2,
		 -2y+9z=31,
		 - y+4z=14}  
	&\begin{array}{l} 
	(a')=(a) \\ 
	(b')=(b)-2(a) \\
	(c')=(c)+3(a)
	\end{array} \\ \Leftrightarrow
	&\systeme{
		x-2y+   z=2,
		 -2y+  9z=31,
		    -\frac{1}{2}z=-\frac{3}{2}}  
	&\begin{array}{l} 
	(a'')=(a') \\ 
	(b'')=(b') \\
	(c'')=(c')-\frac{1}{2}(b')
	\end{array} \\ \Leftrightarrow
	&\begin{cases}
	z=3 \\
	y=\frac{31-9*3}{-2} =-2\\
	x=2+2*(-2)-3=-5
	\end{cases}
	\end{align*}
	\\
	\textbf{Svar}: $(x,y,z)=(-5,-2,3)$
\end{quotation}

\pagebreak
\texttt{\tskcol{1.3~~~~~ (s.)}}
\begin{quotation}
	\noindent
	Gausselimination
	\begin{align*}
		&\systeme{
			  x-2y+ z=1,
			 2x-6y+6z=2,
			-3x+5y+ z=3}  
		&\begin{array}{l} 
			(a) \\ 
			(b) \\
			(c)
		\end{array} \\ \Leftrightarrow
		&\systeme{
			x-2y+ z=1,
			 -2y+4z=0,
			 - y+4z=6}  
		&\begin{array}{l} 
			(a')=(a) \\ 
			(b')=(b)-2(a) \\
			(c')=(c)+3(a)
		\end{array} \\ \Leftrightarrow
		&\systeme{
			x-2y+ z=1,
			 -2y+4z=0,
			     2z=6}  
		&\begin{array}{l} 
			(a'')=(a') \\ 
			(b'')=(b') \\
			(c'')=(c')-\frac{1}{2}(b')
		\end{array} \\ \Leftrightarrow
		&\begin{cases}
			z=3 \\
			y=\frac{-4*3}{-2} =6\\
			x=1+2*6-3=10
		\end{cases}
	\end{align*}
	\\
	\textbf{Svar}: $(x,y,z)=(10,6,3)$
\end{quotation}

\texttt{\tskcol{1.4~~~~~ (s.)}}
\begin{quotation}
	\noindent
	Gausselimination
	\begin{align*}
	&\systeme{
		  x-2y+ z=1,
		 2x-6y+6z=2,
		-3x+5y- z=3}  
	&\begin{array}{l} 
	(a) \\ 
	(b) \\
	(c)
	\end{array} \\ \Leftrightarrow
	&\systeme{
		x-2y+ z=1,
		 -2y+4z=0,
		 - y+2z=6}  
	&\begin{array}{l} 
	(a')=(a) \\ 
	(b')=(b)-2(a) \\
	(c')=(c)+3(a)
	\end{array} \\ \Leftrightarrow
	&\systeme{
		x-2y+ z=1,
		 -2y+4z=0,
		      0z=6}  
	&\begin{array}{l} 
	(a'')=(a') \\ 
	(b'')=(b') \\
	(c'')=(c')-\frac{1}{2}(b')
	\end{array}
	\end{align*}
	Saknar lösning eftersom $0\neq6$.
	\\ \\
	\textbf{Svar}: Lösning saknas
\end{quotation}

\texttt{\tskcol{1.5~~~~~ (s.)}}
\begin{quotation}
	\noindent
	\\ \\
	\textbf{Svar}:
\end{quotation}

\texttt{\tskcol{1.6~~~~~ (s.)}}
\begin{quotation}
	\noindent
	\\ \\
	\textbf{Svar}:
\end{quotation}

\texttt{\tskcol{1.7~~~~~ (s.)}}
\begin{quotation}
	\noindent
	\\ \\
	\textbf{Svar}:
\end{quotation}

\texttt{\tskcol{1.8~~~~~ (s.)}}
\begin{quotation}
	\noindent
	\\ \\
	\textbf{Svar}:
\end{quotation}

\texttt{\tskcol{1.9~~~~~ (s.)}}
\begin{quotation}
	\noindent
	\\ \\
	\textbf{Svar}:
\end{quotation}

\texttt{\tskcol{1.10~~~~ (s.)}}
\begin{quotation}
	\noindent
	\\ \\
	\textbf{Svar}:
\end{quotation}

\texttt{\tskcol{1.11~~~~ (s.)}}
\begin{quotation}
	\noindent
	\\ \\
	\textbf{Svar}:
\end{quotation}

\texttt{\tskcol{1.12~~~~ (s.)}}
\begin{quotation}
	\noindent
	\\ \\
	\textbf{Svar}:
\end{quotation}

\texttt{\tskcol{1.13~~~~ (s.)}}
\begin{quotation}
	\noindent
	\\ \\
	\textbf{Svar}:
\end{quotation}

\texttt{\tskcol{1.14~~~~ (s.)}}
\begin{quotation}
	\noindent
	\\ \\
	\textbf{Svar}:
\end{quotation}

\texttt{\tskcol{1.15~~~~ (s.)}}
\begin{quotation}
	\noindent
	\\ \\
	\textbf{Svar}:
\end{quotation}

\texttt{\tskcol{1.16~~~~ (s.)}}
\begin{quotation}
	\noindent
	\\ \\
	\textbf{Svar}:
\end{quotation}

\texttt{\tskcol{1.17~~~~ (s.)}}
\begin{quotation}
	\noindent
	\\ \\
	\textbf{Svar}:
\end{quotation}

\texttt{\tskcol{1.18~~~~ (s.)}}
\begin{quotation}
	\noindent
	\\ \\
	\textbf{Svar}:
\end{quotation}

\texttt{\tskcol{1.19~~~~ (s.)}}
\begin{quotation}
	\noindent
	\\ \\
	\textbf{Svar}:
\end{quotation}

\texttt{\tskcol{1.20~~~~ (s.)}}
\begin{quotation}
	\noindent
	\\ \\
	\textbf{Svar}:
\end{quotation}

\texttt{\tskcol{1.21~~a) (s.)}}
\begin{quotation}
	\noindent
	\\ \\
	\textbf{Svar}:
\end{quotation}

\texttt{\tskcol{~~~~~~b) (s.)}}
\begin{quotation}
	\noindent
	\\ \\
	\textbf{Svar}:
\end{quotation}

\texttt{\tskcol{1.22~~~~ (s.)}}
\begin{quotation}
	\noindent
	\\ \\
	\textbf{Svar}:
\end{quotation}

\texttt{\tskcol{1.23~~~~ (s.)}}
\begin{quotation}
	\noindent
	\\ \\
	\textbf{Svar}:
\end{quotation}

\texttt{\tskcol{1.24~~~~ (s.)}}
\begin{quotation}
	\noindent
	\\ \\
	\textbf{Svar}:
\end{quotation}

\texttt{\tskcol{1.25~~~~ (s.)}}
\begin{quotation}
	\noindent
	\\ \\
	\textbf{Svar}:
\end{quotation}

\texttt{\tskcol{1.26~~~~ (s.)}}
\begin{quotation}
	\noindent
	\\ \\
	\textbf{Svar}:
\end{quotation}
\end{document}